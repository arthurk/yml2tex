\documentclass[slidestop,red]{beamer}
\usepackage[utf8]{inputenc}
\usepackage[russian]{babel}
\usepackage[T2A]{fontenc}
\usepackage{fancyvrb,color}

\newcommand\PYZat{@}
\newcommand\PYZlb{[}
\newcommand\PYZrb{]}
\newcommand\PYbh[1]{\textcolor[rgb]{0.00,0.50,0.00}{\textbf{#1}}}
\newcommand\PYbg[1]{\textcolor[rgb]{0.73,0.40,0.53}{\textbf{#1}}}
\newcommand\PYbf[1]{\textcolor[rgb]{0.40,0.40,0.40}{#1}}
\newcommand\PYbe[1]{\textcolor[rgb]{0.73,0.13,0.13}{#1}}
\newcommand\PYbd[1]{\textcolor[rgb]{0.00,0.50,0.00}{\textbf{#1}}}
\newcommand\PYbc[1]{\textcolor[rgb]{0.40,0.40,0.40}{#1}}
\newcommand\PYbb[1]{\textcolor[rgb]{0.00,0.00,0.50}{\textbf{#1}}}
\newcommand\PYba[1]{\textcolor[rgb]{0.00,0.50,0.00}{\textbf{#1}}}
\newcommand\PYaJ[1]{\textcolor[rgb]{0.69,0.00,0.25}{#1}}
\newcommand\PYaK[1]{\textcolor[rgb]{0.73,0.13,0.13}{#1}}
\newcommand\PYaH[1]{\textcolor[rgb]{0.10,0.09,0.49}{#1}}
\newcommand\PYaI[1]{\fcolorbox[rgb]{1.00,0.00,0.00}{1,1,1}{#1}}
\newcommand\PYaN[1]{\textcolor[rgb]{0.74,0.48,0.00}{#1}}
\newcommand\PYaO[1]{\textcolor[rgb]{0.00,0.00,1.00}{\textbf{#1}}}
\newcommand\PYaL[1]{\textcolor[rgb]{0.00,0.00,1.00}{#1}}
\newcommand\PYaM[1]{\textcolor[rgb]{0.73,0.73,0.73}{#1}}
\newcommand\PYaB[1]{\textcolor[rgb]{0.00,0.50,0.00}{#1}}
\newcommand\PYaC[1]{\textcolor[rgb]{0.00,0.25,0.82}{#1}}
\newcommand\PYaA[1]{\textcolor[rgb]{0.00,0.63,0.00}{#1}}
\newcommand\PYaF[1]{\textcolor[rgb]{0.63,0.00,0.00}{#1}}
\newcommand\PYaG[1]{\textcolor[rgb]{1.00,0.00,0.00}{#1}}
\newcommand\PYaD[1]{\textcolor[rgb]{0.67,0.13,1.00}{#1}}
\newcommand\PYaE[1]{\textcolor[rgb]{0.25,0.50,0.50}{\textit{#1}}}
\newcommand\PYaZ[1]{\textcolor[rgb]{0.73,0.13,0.13}{#1}}
\newcommand\PYaX[1]{\textcolor[rgb]{0.73,0.13,0.13}{#1}}
\newcommand\PYaY[1]{\textcolor[rgb]{0.00,0.50,0.00}{#1}}
\newcommand\PYaR[1]{\textcolor[rgb]{0.40,0.40,0.40}{#1}}
\newcommand\PYaS[1]{\textcolor[rgb]{0.10,0.09,0.49}{#1}}
\newcommand\PYaP[1]{\textcolor[rgb]{0.00,0.00,0.50}{\textbf{#1}}}
\newcommand\PYaQ[1]{\textcolor[rgb]{0.49,0.56,0.16}{#1}}
\newcommand\PYaV[1]{\textcolor[rgb]{0.82,0.25,0.23}{\textbf{#1}}}
\newcommand\PYaW[1]{\textcolor[rgb]{0.00,0.00,1.00}{\textbf{#1}}}
\newcommand\PYaT[1]{\textcolor[rgb]{0.25,0.50,0.50}{\textit{#1}}}
\newcommand\PYaU[1]{\textcolor[rgb]{0.50,0.00,0.50}{\textbf{#1}}}
\newcommand\PYaj[1]{\textcolor[rgb]{0.10,0.09,0.49}{#1}}
\newcommand\PYak[1]{\textcolor[rgb]{0.25,0.50,0.50}{\textit{#1}}}
\newcommand\PYah[1]{\textcolor[rgb]{0.00,0.50,0.00}{#1}}
\newcommand\PYai[1]{\textcolor[rgb]{0.63,0.63,0.00}{#1}}
\newcommand\PYan[1]{\textbf{#1}}
\newcommand\PYao[1]{\textcolor[rgb]{0.67,0.13,1.00}{\textbf{#1}}}
\newcommand\PYal[1]{\textcolor[rgb]{0.73,0.40,0.53}{#1}}
\newcommand\PYam[1]{\textcolor[rgb]{0.00,0.50,0.00}{\textbf{#1}}}
\newcommand\PYab[1]{\textit{#1}}
\newcommand\PYac[1]{\textcolor[rgb]{0.73,0.13,0.13}{#1}}
\newcommand\PYaa[1]{\textcolor[rgb]{0.50,0.50,0.50}{#1}}
\newcommand\PYaf[1]{\textcolor[rgb]{0.25,0.50,0.50}{\textit{#1}}}
\newcommand\PYag[1]{\textcolor[rgb]{0.40,0.40,0.40}{#1}}
\newcommand\PYad[1]{\textcolor[rgb]{0.73,0.13,0.13}{#1}}
\newcommand\PYae[1]{\textcolor[rgb]{0.40,0.40,0.40}{#1}}
\newcommand\PYaz[1]{\textcolor[rgb]{0.00,0.50,0.00}{\textbf{#1}}}
\newcommand\PYax[1]{\textcolor[rgb]{0.40,0.40,0.40}{#1}}
\newcommand\PYay[1]{\textcolor[rgb]{0.60,0.60,0.60}{\textbf{#1}}}
\newcommand\PYar[1]{\textcolor[rgb]{0.53,0.00,0.00}{#1}}
\newcommand\PYas[1]{\textcolor[rgb]{0.10,0.09,0.49}{#1}}
\newcommand\PYap[1]{\textcolor[rgb]{0.73,0.40,0.13}{\textbf{#1}}}
\newcommand\PYaq[1]{\textcolor[rgb]{0.00,0.50,0.00}{#1}}
\newcommand\PYav[1]{\textcolor[rgb]{0.40,0.40,0.40}{#1}}
\newcommand\PYaw[1]{\textcolor[rgb]{0.00,0.50,0.00}{\textbf{#1}}}
\newcommand\PYat[1]{\textcolor[rgb]{0.73,0.13,0.13}{\textit{#1}}}
\newcommand\PYau[1]{\textcolor[rgb]{0.10,0.09,0.49}{#1}}

\usetheme{Antibes}
\setbeamertemplate{footline}[frame number]
\usecolortheme{lily}
\beamertemplateshadingbackground{blue!5}{yellow!10}

\title{yml2tex (по-русски)}
\author{Юревич Юрий <the.pythy@gmail.com>}
\institute{}
\date{\today}

\begin{document}

\frame{\titlepage}

\section*{Outline}
\frame {
	\frametitle{Outline}
	\tableofcontents
}

\AtBeginSection[] {
	\frame{
		\frametitle{Outline}
		\tableofcontents[currentsection]
	}
}

\section{Введение}
\subsection{Зачем}
\frame {
	\frametitle{Зачем}
	\begin{itemize}[<+-| alert@+>]
	\item Создание презентаций PowerPoint/Keynote и т.д. занимает слишком много времени
	\begin{itemize}[<+-| alert@+>]
	\item Нет автоматической генерации оглавления
	\item Нет подсветки кода
	\item Это не свободное ПО
	\item Проприетарные форматы
	\end{itemize}
	\end{itemize}
}

\section{Возможности}
\subsection{В целом}
\frame {
	\frametitle{В целом}
	\begin{itemize}[<+-| alert@+>]
	\item Легкое и быстрое создание несложных презентаций
	\item Приятный стиль по умолчанию
	\item Генерирует оглавления для разделов/подразделов
	\item Отделение контента от презентации
	\end{itemize}
}
\subsection{Списки}
\frame {
	\frametitle{Ненумерованные списки}
	\begin{itemize}[<+-| alert@+>]
	\item Элемент
	\item Другой элемент
	\begin{itemize}[<+-| alert@+>]
	\item Элемент второго уровня
	\begin{itemize}[<+-| alert@+>]
	\item Элемент третьего уровня
	\item Еще элемент третьего уровня
	\end{itemize}
	\end{itemize}
	\item Следующий элемент
	\item Завершающий элемент
	\end{itemize}
}
\subsection{Подсветка кода}
\begin{frame}[fragile,t]
	\frametitle{Code: "foobar.py"}
\begin{Verbatim}[commandchars=@\[\],numbers=left,firstnumber=1,stepnumber=1]
@PYaE[#!/usr/bin/env python]
@PYaE[# encoding: utf-8]

@PYam[import] @PYaW[sys]
@PYam[import] @PYaW[os]

@PYaz[def] @PYaL[main]():
    @PYaz[pass]

@PYaz[if] __name__ @PYbf[==] @PYad[']@PYad[__main__]@PYad[']:
    main()
\end{Verbatim}


\end{frame}
\frame {
	\frametitle{Исходный код}
	\begin{itemize}[<+-| alert@+>]
	\item Синтаксис
	\begin{itemize}[<+-| alert@+>]
	\item include foobar.py:
	\end{itemize}
	\item Для подсветки используется Pygments
	\item Стиль можно поменять
	\begin{itemize}[<+-| alert@+>]
	\item highlight\_style: colorful
	\end{itemize}
	\end{itemize}
}
\subsection{Картинки}
\frame[shrink] {
	\pgfimage[]{foobar}
}
\frame {
	\frametitle{Картинки}
	\begin{itemize}[<+-| alert@+>]
	\item Синтаксис
	\begin{itemize}[<+-| alert@+>]
	\item image foobar.png:
	\end{itemize}
	\item Необязательные опции
	\begin{itemize}[<+-| alert@+>]
	\item width: 10cm
	\item height: 15cm
	\end{itemize}
	\end{itemize}
}
\end{document}
